%Create titlepage, acknowledgements, abbreviations, toc etc
\thispagestyle{empty}

\begin{titlepage}

\begin{center}

\vspace*{2em}
        \fontsize{25}{24}
        \textbf{An exploration of the sustainability of dinner recipes}
        
        \vspace{0.5cm}
        \large\textbf{Aslaug Angelsen}
        \vspace{0.5cm}\\
		    \includegraphics[width=120mm]{images/emblem}
		    \vspace{0.5cm}\\
		    Master thesis in clinical nutrition\\
		    at the University of Bergen\\
		    \vspace{1.5cm}
		    Supervisor: Christoph Trattner\\
		    Co-supervisor: Alain Starke\\
		    \vspace{0.5cm}
      	\huge\date{2022}
      	
\end{center}
\end{titlepage}
\afterpage{\blankpage}


\let\maketitle\oldmaketitle
%\maketitle

%No headers etc for intro pages
\pagestyle{plain}

% Acknowledgement and abstract sections
\chapter*{Acknowledgements}
Thanks everyone
\pagebreak

\chapter*{Abstract}
\setstretch{1.5}
\textbf{Background:} Our current food system and dietary habits not only contribute to malnutrition and ill health, but also have a damaging environmental impact. A potential source of inspiration for food choice is recipes, but little is known about how well recipes adhere to healthy dietary principles or their environmental sustainability. Here I have explored the healthiness and environmental impact of recipes from three different countries, and the relationship between country of origin, healthiness and environmental impact.\\\\
\textbf{Methods:} Recipes from online recipe sites of Norway (n = 400) and the United States (US, n = 100) and from the United Kingdom (UK) chef's recipe books (n = 100) were included in the analysis. Recipe's healthiness was calculated by comparing their nutrient content to the World Health Organization and the Nordic Nutrition Recommendations for macronutrient intake, the Norwegian recommended daily intake of micronutrients for adult women, and the food labels UK Food Standard Agency multiple traffic light system and the French Nutriscore. The SHARP Indicator database was used to calculate environmental impact. A cross-country analysis was performed by comparing the healthiness scores, nutrient content and environmental impact between countries using the Kruskal Wallis test with a \emph{post-hoc} Dunn's test. Relationship between healthiness and environmental impact was explored by using Spearman's rho to look at correlation between healthiness indicators and environmental impact and the correlation between individual nutrients and environmental impact, and by comparing the environmetal impact of recipes that used foods encouraged in dietary guidelines with those that used foods dietary guidelines recommend to limit.\\\\\\
\textbf{Results:} Small but significant differences (\emph{p}-value <0.05) were found between countries on the healthiness indicators. Recipes from the US scored significantly lower on three out of four healthiness indicators than recipes from the UK, and scored significantly worse on environmental impact than recipes from Norway and the UK (\emph{p}-value <0.05). Recipes scored more favorably on healthiness when assessed with food labels than with the macronutrient criteria. All healthiness indicators were positively correlated with each other (rho >0.4) and negatively correlated with environmental impact (rho <-0.3). Iron and zinc were positively correlated with environmental impact (rho >0.4). The majority of recipes used red meat as a source of protein, with the second most used source of protein being seafood for Norway and the UK and poultry for the US. Few recipes were either vegetarian or vegan, with the majority of vegetarian recipes being from the UK. Recipes that used ruminant meat had higher environmental impact than recipes that used non-ruminant meat, seafood or were plant-based.\\\\
\textbf{Conclusion:} The type of healthiness indicator used can influence if a recipe is classified as healthy or not, while there were small but significant differences between recipe healthiness, environmental impact and country of origin. Regardless of healthiness indicator used healthier recipes had lower environmental impact, but lower environmental impact also meant a reduction in important nutrients iron and zinc. Despite dietary guidelines recommending that red meat intake should be limited, while simultaneously encouraging the intake of seafood or plant-based food, red meat was the most used protein source. Recipes with red meat from ruminants also had the highest environmental impact.
\pagebreak
